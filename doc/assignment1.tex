\documentclass{article}
\usepackage{a4wide}
\usepackage{hyperref}
%\setlength{\parindent}{0pt}
\title{Web Data Management, assignment 1}
\author{Hans van den Bogert\\1307983 \and Bastiaan van Graafeiland\\1399101}
\begin{document}
\maketitle

\section{XPath assignments}
We will have to device queries for both the movies dataset as well as the movies\_refs dataset. The latter has actors and movies seperated.

\subsection{All title elements}
For both the data sets, titles can be retrieved by:
\begin{verbatim}
//title
\end{verbatim}

\subsection{All movie title (i.e. the textual value of title element)}
Every node has a text representation.
\begin{verbatim}
//title/text()
\end{verbatim}

\subsection{Titles of the movies published after 2000}
Assuming 'after 2000' means 2001 and up, the following added predicate will work on both xml files:
\begin{verbatim}
/movies/movie[year>2000]/title
\end{verbatim}

\subsection{Summary of "Spiderman"}
The movie named "Spiderman" does not exist. Assuming "Spider-Man" was meant, the following query will provide the movie's summary:
\begin{verbatim}
/movies/movie[title='Spider-Man']/summary
\end{verbatim}

\subsection{Who is the director of Heat}
\subsubsection{movies.xml}
\begin{verbatim}
/movies/movie[title='Heat']/director
\end{verbatim}
\subsubsection{movies\_refs.xml}
We need to reference the id we get in the case of the \emph{movies\_refs.xml} file.
\begin{verbatim}
//artist[@id=//movie[title='Heat']/director/@id]
\end{verbatim}
% /movies/artist[@id=string(doc('/db/movies/movies\_refs.xml')/movies/movie[title/text()='Heat']/director/@id)]

\subsection{Title of the movies featuring Kirsten Dunst}
\subsubsection{movies.xml}
\begin{verbatim}
//movie[actor/first\_name='Kirsten'][actor/last_name='Dunst']/title
\end{verbatim}
\subsubsection{movies\_refs.xml}
\begin{verbatim}
//movie[actor/@id = //artist[last_name='Dunst' and first_name='Kirsten']/@id]
	/title
\end{verbatim}

\subsection{Which movies have a summary}
\begin{verbatim}
//movie[summary]
\end{verbatim}

\subsection{Which movies do not have a summary}
\begin{verbatim}
//movie[not(summary)]
\end{verbatim}
\texttt{count()} could have been used as well, counting the number of \emph{summary} elements.

\subsection{Titles of the movies published more than 5 years ago}
The following query gets the current date, extracts the year, substracts 5 years and then uses that year as a predicate.
\begin{verbatim}
//movie[year < (year-from-dateTime(current-dateTime()) - 5)]/title
\end{verbatim}

\subsection{What was the role of Clint Eastwood in Unforgiven?}
\subsubsection{movies.xml}
\begin{verbatim}
//movie[title='Unforgiven']
	/actor[first_name='Clint' and last_name='Eastwood']/role
\end{verbatim}
\subsubsection{movies\_refs.xml}
\begin{verbatim}
//movie[title='Unforgiven']
	/actor[@id = //artist[first_name='Clint' and last_name='Eastwood']/@id]
	/@role/string()
\end{verbatim}
The result differs with the \emph{movies.xml} query, where the result was a node rather than a string. This is inherent to the structure changes where role is either an attribute or a node itself. The assignment was not specific in the desired result.

\subsection{What is the last movie of the document}
\begin{verbatim}
//movie[last()]
\end{verbatim}

\subsection{Title of the film that immediately precedes Marie Antoinette in the document}
In both sets:
\begin{verbatim}
//movie[title='Marie Antoinette']/preceding-sibling::movie[1]/title
\end{verbatim}
The first element in the set of preceding siblings has to be obtained, because the axis is backwards.

\subsection{Get the movies whose title contains "V"}
Restricted to uppercase "V", the query for both data-sets:
\begin{verbatim}
//movie[contains(title, 'V')]
\end{verbatim}

\subsection{Get the movies whose cast consists of exactly three actors}
\begin{verbatim}
//movie[count(actor) = 3]
\end{verbatim}

\section{XQuery assignments}
For all queries these lines are implicit:
\begin{verbatim}
let $ms:=doc("movies-query/movies_alone.xml"),
		$as:=doc("movies-query/artists_alone.xml")
\end{verbatim}

\subsection{List the movies published after 2002, including their title and year}
The assignment was not entirely clear on the expected output, so it's just a string.
\begin{verbatim}
for $m in $ms/movies/movie
where year > 2002
return concat($m/title/text(), ', ', $m/year/text())
\end{verbatim}

\subsection{Create a flat list of all the title-role pairs, with each pair enclosed in a \emph{result}� element. Here is an example of the expected result structure:...}
\begin{verbatim}
<results>
{
    for $role in $ms/movies/movie/actor/@role
    return <result>
    	{$role/../../title}
    	<role>{$role/string()}</role>
    	</result>
}
</results>
\end{verbatim}

\subsection{ Give the title of movies where the director is also one of the actors.}
\begin{verbatim}
for $m in $ms//movie
where director/@id = actor/@id
return $m/title
\end{verbatim}

\subsection{ Show the movies, grouped by genre. Hint: function distinct-values() removes the duplicates from a sequence. It returns atomic values.}
The requested format is unclear, we opted for a list of tuples with the return type (genre, corresponding movie title)
\begin{verbatim}
for $g in distinct-values($ms//genre)
return ($g,$ms//movie[genre=$g]/title)
\end{verbatim}
Alternatively, \texttt{group by} could be used:
\begin{verbatim}
for $m in $ms//movie
let $g := $m/genre
group by $g
return ($g, $m/title)
\end{verbatim}

\subsection{For each distinct actor's id in movies\_alone.xml, show the titles of the movies where this actor plays a role. The format of the result should be:...}
\begin{verbatim}
for $id in distinct-values($ms//actor/@id)
let $movies := $ms//movie[actor/@id=$id]
return <actor>{$id, $movies/title} </actor>
\end{verbatim}

\subsubsection{Variant}
The variant asks only for the nodes/actors which appear in at least 2 movies. This can be achieved by adding a \texttt{where} clause to the query:
\begin{verbatim}
for $id in distinct-values($ms//actor/@id)
let $movies := $ms//movie[actor/@id=$id]
where count($movies) > 1
return <actor>{$id, $movies/title} </actor>
\end{verbatim}

\subsection{Give the title of each movie, along with the name of its director. Note: this is a join!}
The result is a concatenation of the movie title and the director's name.
\begin{verbatim}
for $m in $ms//movie, $a in $as//artist
where $m/director/@id = $a/@id
return concat($m/title, ": ", $a/last_name, ', ', $a/first_name)
\end{verbatim}

\subsection{Give the title of each movie, and a nested element <actors> giving the list of actors with their role.}
Again the desired output is very ambiguous and left open for interpretation, we read it as: (1) sequence the titles as elements, in these titles include (nest) an extra element actors which contains elements for the actor's name and role.
\begin{verbatim}
for $m in $ms//movie
let $actors := $as//artist[@id = $m/actor/@id]
return <title>{$m/title/text()}<actors>{
    for $a in $actors
    let $role := $m/actor[@id=$a/@id]/@role/string()
    return <actor>
        <name>{concat($a/first_name, " ", $a/last_name)}</name>
        <role>{$role}</role>
        </actor>
}</actors></title>
\end{verbatim}
Another interpretation would be to have the role as an attribute:
\begin{verbatim}
for $m in $ms//movie return 
<title>{$m//title/text()}
    <actors>
    {for $ac in $m/actor, $ar in $as//artist
    where $ac/@id = $ar/@id
    return 
        <actor role="{$ac/@role/string()}">                                   
            {$ar/first_name/text()}\&#032;{$ar/last_name/text()}
        </actor>}
    </actors>
</title>
\end{verbatim}

\subsection{ For each movie that has at least two actors, list the title and first two actors, and an empty "et-al" element if the movie has additional actors. For instance: (...)}
\begin{verbatim}
for $m in $ms//movie
let $actors := $as//artist[@id = $m/actor/@id]
where count($actors) >= 2
return 
		<result>{
			$m/title,
	    for $a in subsequence($actors, 1, 2)
	    let $role := $m/actor[@id=$a/@id]/@role/string()
	    return
	    	<actor>
	    		{concat($a/first_name, ' ', $a/last_name), ' as ', $role}
	    	</actor>,
	    if (count($actors) > 2) then <et-al/> else ''}
    </result>
\end{verbatim}

\subsection{List the titles and years of all movies directed by Clint Eastwood after 1990, in alphabetic order.}
The output is a \emph{result} element, containing the title and year as nested elements.
\begin{verbatim}
let $eastwoodId := $as//artist[last_name="Eastwood"]/@id/string()
for $m in $ms//movie
where $m/director/@id/string() = $eastwoodId and $m/year > 1990
order by $m/title ascending
return <result>{$m/title, $m/year}</result>
\end{verbatim}

\section{Project: Getting started}
We have used the \emph{movies.xml} file as backend. To mimic a
real-life setting, where a person would use a xml-backend from a
3rd-party in a web application, we've chosen to use javascript as the
means to retrieve the xml-data. The data-flow is as follows:
\begin{enumerate}
\item the user opens index.php on server '1'
\item user can choose filters, and thereby triggers xquery calls to
  server '1' (using javascript)
\item server '1' then proxies the request to server '2', the
  exist-db. (using php -- all this to circumvent the browser
  cross-origin restriction)
\item server '2' executes the query and gives results as XML
\item the user receives that XML through the proxy server '1' and then
  translates this to a list of HTML using jQuery.
\end{enumerate}

The benefit of using javascript is that the results can be fetched
on-the-fly using the capabilities of modern browsers.

\section{Shakespeare application}
The assignment was not clear on which collection should be used, there
were 2 choices: (1) The 3 works of Shakespeare in the samples
directory actually \emph{shipped} with the eXist-db, or (2) the
application downloadable by the package manager in eXist-db, we chose
the former.  Some parts we're reused from the samples directory,
namely the xsl file.

The application is built using the application-framework of eXist-db
much like the Shakespeare application demo, although we've chosen to
use xslt to do most of the transformations instead of using xquery.

The following lists depicts the step taken from a request to the
exist-db by the exist-db to render the page.
\begin{enumerate}
\item the user connects to the exist-db which has the application
  installed
  e.g. \url{http://exist-db:8080/exist/apps/shakespeare/index.html}
\item The controller.xql receives the URL request and routes it
  accordingly - usually to view.xql
\item the view.xql looks at the original request and finds the
  appropriate html file, like index.html, view-play.html etc.
\item The view.xql then parses this HTML-snippet to find directives on
  where it should be placed in another template HTML file -- located
  in the templates/ directory of the application.
\item The combined HTML is then parsed for directives again for calls
  to the app.xql and places the result in the HTML element calling the
  xquery function.
\item In our case the app.xql will call a transformation function with
  the appropriate stylesheet and return the output into the HTML view.
\end{enumerate}
\paragraph{Notes}
There were ambiguous or vague matters in the assignment, which
collection to use was already addressed in the beginning of this
section, but there were x more issues: (1) What is meant by a
character's \emph{part} for a given scene/act was not defined. Nowhere
in the XML-files is a character's part described in any other way than
just his/her lines -- though it seemed odd to list those, we do so,
and the act/scene can be chosen from the table of contents.

Another (2) issue is that it is unclear where in the XML-files the
\emph{stage's requirements} are defined. (This holds for both the
types of Shakespeare collections referred to in the beginning of this
section.) For now we have listed all the stage directions, which do
include the items and such.

\section{MusicXML online}
The book leaves a lot of the requirements for this project quite open. In the end, we decided to create a web application that allows the user to enter one word as a search term.

\subsection{Resulting work}
The code relevant to this application can be found in the following files:
\begin{itemize}
	\item \texttt{query.xq}
	\item \texttt{application/controllers/welcome.php}
	\item \texttt{application/views/search\_page.php}
\end{itemize}
The music is stored in the music-xml folder. Furthermore we used Twitter Bootstrap for the stylesheets, and CodeIgniter as PHP framework.

\subsection{The application}
The application consists of two parts: the Exist-db server, where the XML files are stored, as well as an XQuery file. This file contains the logic for searching the music database. The search results are returned as \emph{result} elements, where the title of the piece, the composer, and the URI of the XML file are nested. The query works as follows: in each file of the collection (\emph{score-partwise} element), the \emph{movement-title, creator}, and \emph{credit-words} elements are checked for the query word. If the word is contained in any of these elements, the piece is considered a match.

The other part is the web application itself, written in PHP using the CodeIgniter MVC framework. On the search page, a form is displayed where the search term can be entered. All music sheets containing this word in their metadata are returned, and the user can then download PDF or MIDI versions of the music. This works as follows: upon clicking a search result (PDF or MIDI), the Exist server is queried for the corresponding XML file. This file is then converted to a Lilypond file, which is in turn converted to PDF and MIDI. This is done only once per file, so subsequent downloads can be provided immediately without waiting for the conversion process.

In short, the following steps are taken when searching for and downloading a music sheet.
\begin{enumerate}
	\item The user enters a search term and clicks on Search
	\item The web app delegates the search term to the XQuery file on the Exist database
	\item Using XQuery, different child elements of each piece are searched for a match
	\item The search results are obtained by the web app as an XML file, which is then parsed
	\item Using said XML file, a list of search results is generated along with links to PDF and MIDI versions
	\item The user clicks the PDF/MIDI link of a search result
	\item The web app checks whether the requested file already exists, if so, it is returned immediately
	\item Else, the MusicXML file for the requested piece is retrieved from the Exist server
	\item Using Lilypond, the file is converted to both PDF and MIDI (-m flag with \texttt{musicxml2ly})
	\item The requested file is presented to the user
\end{enumerate}

\subsection{Issues}
Although MusicXML is a standard format designed to work with lots of different software, the exact details can still differ significantly. For instance, some of the resulting XML files did not contain any \emph{composer} or \emph{movement-title} nodes, but the composer and title were stored in \emph{credit-words} elements. If a system akin to this one is to be used in practice, consistency among different MusicXML files is required. The \emph{credit-words} elements are not quite suitable, because they often do not contain any information regarding the semantics of their content (in some cases, they have a \emph{credit-type} sibling element though).

Furthermore, lyrics in music are quite hard to perform a search on, because the words are split up in the XML file; often only syllables are stored in a single element, because this way they can each be put under the correct note in the sheet music.

\subsection{Future work}
It would be nice to be able to search in lyrics. To this end, all the lyric elements could be concatenated, then split up into proper sentences using other software.

Entering your own music was mentioned in the assignment, however, we see no need for this. There are already plenty of evolved software packages out there which have been developed for this. Most of these can export sheets to the MusicXML format as well, so own input would mean reinventing the wheel.

Other search methods were too complex to implement, such as whistling a song into the microphone. This is a good idea, but we believe it is beyond the scope of this exercise.

In general, to implement more advanced features, the MusicXML files should be studied more in depth.

\subsection{Music sources}
Most of the MusicXML sources were obtained by exporting sheets from \href{http://www.ninsheetm.us}{NinSheetMusic}, furthermore we used \href{http://www.musescore.com}{MuseScore} and \href{http://www.openmusicscore.org}{Open Music Score}.

\end{document}
