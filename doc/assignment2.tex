\documentclass{article}
\usepackage{a4wide}
\usepackage{hyperref}
%\setlength{\parindent}{0pt}
\title{Web Data Management, assignment 2}
\author{Hans van den Bogert\\1307983 \and Bastiaan van Graafeiland\\1399101}
\begin{document}
\maketitle

\section{Introduction}
This document describes the choices made for assignment 2 which in our
case was chapter 19 from: "Web Data Management" by Serge
Abiteboul et al.

\section{Exercise 19.5.1 }
For this exercise we basically copied the source-code from the example
(as was implied by the exercise) and added the combiner function of
\texttt{Authors.CountReducer}.

\section{Exercise 19.5.2}
Normally hadoop by default works on a granularity of lines for the
mappers. In the case of XML-data this does not work. Fortunately a lot
of other persons have already solved this by using hadoops extensible
\texttt{InputFormat}. In particular we used the class
\texttt{XMLInputFormat} from the \emph{Cloud9} hadoop toolkit, which
only needed minor alterations to work in our setting. Now by using
this input formatter, the mappers work on the granularity of enclosing
\texttt{movie} tags. 

\end{document}
